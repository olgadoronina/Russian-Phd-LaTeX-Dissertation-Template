
{\actuality} \par{\textbf{Актуальность темы исследования:}} 
Одной из наиболее сложных задач вычислительной газовой динамики и аэродинамики является математическое моделирование течений вокруг движущихся тел.
В частности,  большое количество фундаментальных задач направлено на исследование  аэродинамики движущихся относительно друг друга тел и/или элементов одного тела.
К таким задачам можно отнести исследование аэродинамических и акустических свойств крыла самолёта с механизацией, включая выдвигающиеся или задвигающиеся закрылки и предкрылки; влияние на аэродинамику и акустику открывающихся технологических отсеков и выдвигаемых частей, например, при выпускании шасси самолёта; моделирование траектории сбрасываемых грузов и ряд других проблем. 

В таких задачах могут присутствовать два типа тел, движущихся относительно друг друга. Первый тип – это основное, родительское, тело, второй – тело или тела, совершающие, вообще говоря, достаточно сложные движения относительно основного тела. В такой ситуации трудно, а порою даже невозможно, использовать традиционный подход к описанию обтекаемого объекта граничными узлами или граничными элементами расчётной сетки. Такой подход логично применить только к покоящемуся родительскому телу, в системе расчёта которого моделировать передвижения остальных, более мелких, объектов. 

Для описания движущихся тел в качестве наиболее удобного подхода все большее распространение приобретают методы погруженных границ (в англоязычной научной литературе – Immersed Boundary Conditions или IBC метод)~\cite{mittal2005immersed, boiron2009high, brown2014characteristic}.
Методы погруженных границ составляют целый класс методов, позволяющих обеспечить выполнение граничных условий на поверхности обтекаемых тел в процессе численного расчёта без описания его геометрии сеточными узлами. Это означает, что может быть использована «сплошная» расчётная сетка, покрывающая всю область определения задачи, включая препятствие. 

Как и в случае неподвижных тел, повысить точность моделирования перемещающихся объектов и образующихся вокруг них пограничных слоёв в рамках метода погруженных границ можно за счёт сгущения сетки в областях, прилегающих к погруженным границам таких объектов. Возможным, хотя далеко не оптимальным способом  организации нужного сгущения является использование стационарных сеток с достаточно мелкими сеточными элементами в подобласти, покрывающей траекторию возможного движения моделируемого тела. Понятно, что для продвижения на сколь-нибудь дальнее расстояние этот метод является весьма трудоёмким, т.к. ведёт к необходимости использования огромных расчётных сеток. К тому же он не даёт решения в том случае, когда траектория движения тела неизвестна заранее, а определяется в процессе расчёта на основе вычисления действующих на тело аэродинамических сил. 

Выходом в такой ситуации может стать использование динамической адаптации сетки, причём такой, которая была бы заметно дешевле по стоимости расчёта на изначально всюду очень подробных стационарных сетках и не приводила к ухудшению параллельной эффективности всего численного алгоритма. 

В данной работе рассматривается подход, основанный на преобразовании координат узлов или «подвижной сетке», и его адаптация к алгоритму моделирования течения вокруг подвижных твёрдых тел, заданных погруженными границами. Работоспособность разработанных алгоритмов демонстрируется на решении двумерных тестовых задач на неструктурированных треугольных сетках при использовании упрощённой формулировки метода.

{\aim} данной работы является разработка эффективной технологии динамической адаптации неструктурированных сеток к границе движущегося обтекаемого тела заданного методом погруженных граничных условий. Для~достижения поставленной цели необходимо было решить следующие {\tasks}:
\begin{enumerate}
	\item Анализ существующих алгоритмов динамической сеточной адаптации.
  \item  Разработка алгоритма динамической адаптации неструктурированной сетки. 
   \item Адаптация алгоритма к использованию погруженных граничных условий и функции линий уровня, описывающей геометрию твёрдого тела.
  \item Реализация алгоритма в рамках некоммерческого исследовательского комплекса программ NOISEtte++, разрабатываемом в отделе вычислительной аэродинамики и аэроакустики ИПМ РАН.
  \item Рассмотрение двумерной постановки для треугольных сеток. Тестирование на тестовых задачах с аналитически заданной геометрией тела и заданным законом его движения.
\end{enumerate}

{\novelty}
Новаторство предлагаемых в работе решений заключается в первую очередь в применении метода «быстрой», не меняющей топологии, динамической адаптации сетки в комплексе с заданием геометрии обтекаемого объекта погруженными граничными условиями. 

{\influence} Данный подход к адаптации сетки позволяет создание более эффективной методики численного исследования движущихся в турбулентных потоках тел по сравнению с существующими подходами, использующими неструктурированные сетки.  Помимо использования алгоритма адаптации совместно в погруженными граничными условиями, его можно успешно использоваться для решения задач с «традиционным» описанием подвижных тел граничными сеточными узлами при условии не столь
больших отклонений от их исходного положения (например, разного рода 
колебательных движений), а также, для адаптации сетки к полям решения.

{\methods} Методы работы основаны на численном решении систем линейных алгебраических уравнений реализованном в программном комплексе 
 NOISEtte++ с использованием MPI-OpenMP распараллеливания. Также при решение тестовых задач использовались методы и модели газовой динамики реализованные в программном комплексе NOISEtte++, в частности численные  схемы с квазиодномерной реконструкцией переменных~\cite{abalakin2016edge}, методы поглуженных границ~\cite{abalakin2019ibm, kozubskaya2019numerical} и описание аэродинамики движущихся тел на основе метода погруженных границ на подвижной сетке (см. Приложение~\ref{app:A}). Результаты тестовых расчетов сравниваются с данными эксперимента.

{\defpositions}
\begin{enumerate}
\item  Разработан алгоритм динамической адаптации неструктурированной сетки для  использования совместно в методом погруженных граничных условий и функции линий уровня, описывающей геометрию твёрдого тела.
\item Алгоритм реализаван в рамках некоммерческого исследовательского комплекса программ NOISEtte++ с использованием MPI-OpenMP распараллеливания, разрабатываемом в отделе вычислительной аэродинамики и аэроакустики ИПМ РАН.
\item На основе разработанного алгоритма проведен вычислительный эксперимент моделирования течения за движущимся цилиндром.
\end{enumerate}

{\reliability} разработанных алгоритмов демонстрируется на решении двумерных тестовых задач на неструктурированных треугольных сетках. Полученные результаты сравниваются с экспериментальными данными, полученными другими авторами.

{\probation}
Основные результаты работы докладывались~на:
\begin{enumerate}
	\item Доклад на 14-ом международном научном семинаре «Математические модели и моделирование в лазерно-плазменных процессах и передовых научных технологиях» (LPPM3-2016). Тема доклада:  «Динамическая адаптация треугольной сетки к границе движущегося объекта, заданного с использованием метода погруженных границ, на основе алгоритма перераспределения узлов».

	\item Доклад на 6-ой Всероссийской конференции «Вычислительный эксперимент в аэроакустике» (г. Светлогорск Калининградской области, 19-24 сентября 2016 года). Тема доклада: «Моделирование аэродинамики движущихся тем с использованием сеточной адаптации в погруженным границам на неструктурированных треугольных сетках».~\citeauthor{ceaa_2016}
\end{enumerate}

{\publications} Основные результаты по теме диссертации опубликованы в 1 работе в издании, входящем в базу Scopus:
\begin{enumerate}
	\item Статья ``Моделирование аэродинамики движущегося тела, заданного погруженными границами на динамической адаптивной неструктурированной сетке`` в журнале ``Математическое моделирование``~\citeauthor{abalakin2018russian}.\\
	Статья переведена на английский язык: ``Simulating Aerodynanics of a Moving Body Specified by Immersed Boundaries on Dynamically Adaptive Unstructured Meshes`` в журнале ``Mathematical Models and Computer Simulations``~\citeauthor{abalakin2019simulating}
\end{enumerate}
