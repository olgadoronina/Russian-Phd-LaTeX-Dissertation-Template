%% Согласно ГОСТ Р 7.0.11-2011:
%% 5.3.3 В заключении диссертации излагают итоги выполненного исследования, рекомендации, перспективы дальнейшей разработки темы.
%% 9.2.3 В заключении автореферата диссертации излагают итоги данного исследования, рекомендации и перспективы дальнейшей разработки темы.
\begin{enumerate}
	\item На основе анализа существующих методов динамической сеточной адаптации выбран метод адаптации с движением узлов без изменения топологии сетки.
	\item  Разработан алгоритм динамической адаптации неструктурированной сетки для  использования совместно в методом погруженных граничных условий и функции линий уровня, описывающей геометрию твёрдого тела.
	\item Алгоритм реализаван в рамках некоммерческого исследовательского комплекса программ NOISEtte++, разрабатываемом в отделе вычислительной аэродинамики и аэроакустики ИПМ РАН.
	\item Рассмотренна двумерная постановка для треугольных сеток. Проведен численный расчет задаче о моделировании течения за движущимся цилиндром.
\end{enumerate}
