\chapter*{Введение}                         % Заголовок
\addcontentsline{toc}{chapter}{Введение}    % Добавляем его в оглавление
	Одной из наиболее сложных задач вычислительной газовой динамики и аэродинамики, в частности, является математическое моделирование течений вокруг движущихся тел. Численное предсказание возникающих таким образом сложных взаимосвязанных газодинамических явлений может внести существенный вклад в решение большого числа фундаментальных проблем в аэродинамике систем перемещающихся тел и/или движущихся элементов одного тела. К таким задачам можно отнести исследование аэродинамических и акустических свойств крыла самолёта с механизацией, включая выдвигающиеся или задвигающиеся закрылки и предкрылки; влияние на аэродинамику и акустику открывающихся технологических отсеков и выдвигаемых частей, например, при выпускании шасси самолёта; моделирование траектории сбрасываемых грузов и ряд других проблем. В таких задачах могут присутствовать два типа тел, движущихся относительно друг друга. Первый тип – это основное, родительское, тело, второй – тело или тела, совершающие, вообще говоря, достаточно сложные движения относительно основного тела. В такой ситуации трудно, а порою даже невозможно, использовать традиционный подход к описанию обтекаемого объекта граничными узлами или граничными элементами расчётной сетки. Такой подход логично применить только к покоящемуся родительскому телу, в системе расчёта которого моделировать передвижения остальных, более мелких, объектов. Для описания движущихся тел в качестве наиболее удобного подхода все большее распространение приобретают методы погруженных границ (в англоязычной научной литературе – Immersed Boundary Conditions или IBC метод)~\cite{mittal2005immersed, boiron2009high, brown2014characteristic}.

	Методы погруженных границ составляют целый класс методов, позволяющих обеспечить выполнение граничных условий на поверхности обтекаемых тел в процессе численного расчёта без описания ее геометрии сеточными узлами. Это означает, что может быть использована «сплошная» расчётная сетка, покрывающая всю область определения задачи, включая препятствие. Взаимодействие твёрдого тела и окружающей среды при этом может моделироваться несколькими способами. Один из них – это модификация дискретизированных газодинамических уравнений, основанная на геометрическом анализе пространственного расположения используемого разностного шаблона относительно границы твёрдого тела. Другой способ заключается во введении штрафных функций непосредственно в исходные уравнения. Они добавляются в виде источниковых членов и имеют ненулевое значение только в узлах расчётной сетки, находящихся внутри твёрдого тела.

	Применение погруженных граничных условий заметно упрощает воспроизведение движения тел даже, если они имеют сложную геометрическую форму. Так, при использовании штрафных функций, изменение пространственного положения обтекаемого тела на каждом временном шаге сводится лишь к изменению источниковых членов в решаемых уравнениях, а не к ресурсоемкому перестроению расчётной сетки, требуемому при описании границы сеточными узлами.

	Как и в случае неподвижных тел, повысить точность моделирования перемещающихся объектов и образующихся вокруг них пограничных слоёв в рамках метода погруженных границ можно за счёт сгущения сетки в областях, прилегающих к погруженным границам таких объектов. Возможным, хотя далеко не оптимальным способом  организации нужного сгущения, является использование стационарных сеток с достаточно мелкими сеточными элементами в подобласти, покрывающей траекторию возможного движения моделируемого тела. Понятно, что для продвижения на сколь-нибудь дальнее расстояние это путь является весьма трудоёмким, т.к. ведёт к необходимости использования огромных расчётных сеток. К тому же, он не даёт решения в том случае, когда заранее траектория движения тела неизвестна, а определяется в процессе расчёта на основе вычисления действующих на тело аэродинамических сил. Выходом в такой ситуации может стать использование динамической адаптации сетки, причём такой, которая была бы заметно дешевле по стоимости расчёта на изначально всюду очень подробных стационарных сетках и не приводила к ухудшению параллельной эффективности всего численного алгоритма. 
	
	Подробные обзоры методов динамической адаптации сетки представлены в книгах Томпсона~\cite{thompson_handbook_1998}, Лисейкина~\cite{liseikin2009grid}, Хуанг и Рассела~\cite{huang_adaptive_2011}. В целом, все техники адаптации могут быть разделены на две большие группы: $h$-методы которые основаны на локальном измельчении и огрублении сетки, и $r$-методы, которые предполагают передвижение узлов сеточных элементов во всей сетке (или на ее обширном участке), чтобы обеспечить желаемую концентрацию в нужных регионах. В методах локального сеточного измельчения/огрубления ($h$-методы)~\cite{adjerid_high-order_1992,eriksson_adaptive_1991,babuska_p-and_1990} узлы добавляются или, наоборот, удаляются на фиксированных временных слоях так, чтобы численное решение на них разрешалось с заданной точностью. Подход, основанный на преобразовании координат узлов или подвижной сетке во всей расчётной области или в достаточно большой подобласти ($r$-методы)~\cite{huang_adaptive_2011}, в основном, используется для динамической адаптации сетки к численному решению путём стягивания сеточных узлов в области с большими градиентами. Такая техника успешно используется во всем мире при решении широкого круга прикладных задач и, в том числе, в вычислительной газовой динамике~\cite{baker_mesh_1997,breslavskii_dynamic_2008,yanenko_methods_1976}. Методы подвижной сетки могут быть также разделены на характерные группы алгоритмов, а именно на вариационные методы и методы, основанные на решении дифференциальных уравнений.
	
	Согласно вариационному подходу, преобразование координат узлов определяется путём минимизации некоторого сеточного функционала. В работах разных авторов предлагаются различные типы подходящих сеточных функционалов. Так, в работе Винслоу (Winslow)~\cite{winslow_adaptive-mesh_1981} представлен эквипотенциальный метод, основанный на диффузии переменных. Вариационный метод Брэкбилла и Зальцмана (Brackbill, Saltzman)~\cite{brackbill_adaptive_1982}  строится на минимизации функции, зависящей от концентрации сеточных узлов, сеточной гладкости и ортогональности. В статье~\cite{dvinsky_adaptive_1991}  предлагается сеточный функционал, зависящий от энергии гармонических отображений, а работах Кнаппа (Knupp) с соавторами \cite{knupp_jacobian-weighted_1996, knupp_framework_2000}  используется идея обусловленности Якобиана преобразования координат. В \cite{huang_variational_2001}  минимизируется функционал, реализующий условия так называемой «равнораспределённости». Наиболее свежий обзор сеточных функционалов, которые могут использоваться с целью адаптации, и их сравнение даётся в работе \cite{huang_comparative_2015}.
	
	В методах, основанных на дифференциальных уравнениях подвижной сетки (MMPDE – moving mesh partial differential equations)~\cite{huang_moving_1994}, уравнение движения сетки и система газодинамических уравнений часто решаются совместно. В \cite{huang_moving_1998, huang_analysis_1997, huang_high_1998} уравнения движения сетки формулируются в зависимости от градиентов моделируемого течения применительно к двумерным задачам. Практические вопросы формулировки уравнений движения сетки и их решения подробно рассматриваются в работах \cite{huang_practical_2001} и \cite{budd_adaptivity_2009}.
	
	Как в вариационных, так и в MMPDE методах расположением сеточных узлов управляет векторная или скалярная функция-монитор. Функция мониторинга обычно строится с целью оценки ошибки численного решения и равномерного распределения её по всем сеточным узлам~\cite{cao_study_1999}.
	
	К классу MMPDE методов можно отнести и несколько более упрощённый подход к адаптации сетки, который предложен Tukovic и Jasak~\cite{jasak_automatic_2006, tukovic_moving_2012} и реализован в открытой библиотеке OpenFOAM~\cite{jasak_dynamic_2010}. В этом способе перемещения узлов сетки определяется уравнением Лапласа с переменной коэффициента диффузии, выполняющего роль функции-монитора. Авторы показали успешное применение этого метода при численном моделировании многофазных потоков.
	
	В целом, MMPDE методы нам представляются наиболее эффективным подходом к динамической сеточной адаптации в задачах моделирования течений вокруг движущихся тел, если эти тела описываются при помощи метода погруженных границ. Использование погруженных границ и, благодаря этому, проведение расчётов подвижных объектов в односвязных областях открывает возможность расширения области применения MMPDE методов, которые до сих пор использовались, в основном, для адаптации сетки к полям численного решения, на задачи, требующие адаптации к поверхности движущегося твёрдого тела. Привлекательность именно методов, основанные на решении уравнений движения сетки, т.е. MMPDE методов, для этого класса задач объясняется несколькими причинами. Во-первых, они сохраняют топологию исходной сетки и не требуют интерполяции решения в новые узлы на каждом временном слое. Во-вторых, сохранение сеточной топологии и отсутствие новых узлов позволяют использовать одну и ту же структуру данных и избежать динамического перераспределения узлов по подобластям декомпозиции при расчёте на параллельных вычислительных системах. Такой подход к моделированию аэродинамики движущихся тел на неструктурированных сетках позволяет существенно сократить вычислительные затраты.
	
	В данной работе рассматривается общая и упрощённая формулировки MMPDE метода сеточной адаптации и разрабатывается численный алгоритм моделирования течения вокруг подвижных твёрдых тел, моделируемых погруженными границами, на его основе. Работоспособность разработанных алгоритмов демонстрируется на решении двумерных тестовых задач на неструктурированных треугольных сетках при использовании упрощённой формулировки метода.

\newcommand{\actuality}{}
\newcommand{\progress}{}
\newcommand{\aim}{{\textbf\aimTXT}}
\newcommand{\tasks}{\textbf{\tasksTXT}}
\newcommand{\novelty}{\textbf{\noveltyTXT}}
\newcommand{\influence}{\textbf{\influenceTXT}}
\newcommand{\methods}{\textbf{\methodsTXT}}
\newcommand{\defpositions}{\textbf{\defpositionsTXT}}
\newcommand{\reliability}{\textbf{\reliabilityTXT}}
\newcommand{\probation}{\textbf{\probationTXT}}
\newcommand{\contribution}{\textbf{\contributionTXT}}
\newcommand{\publications}{\textbf{\publicationsTXT}}


{\actuality} \par{\textbf{Актуальность темы исследования:}} 
Одной из наиболее сложных задач вычислительной газовой динамики и аэродинамики является математическое моделирование течений вокруг движущихся тел.
В частности,  большое количество фундаментальных задач направлено на исследование  аэродинамики движущихся относительно друг друга тел и/или элементов одного тела.
К таким задачам можно отнести исследование аэродинамических и акустических свойств крыла самолёта с механизацией, включая выдвигающиеся или задвигающиеся закрылки и предкрылки; влияние на аэродинамику и акустику открывающихся технологических отсеков и выдвигаемых частей, например, при выпускании шасси самолёта; моделирование траектории сбрасываемых грузов и ряд других проблем. 

В таких задачах могут присутствовать два типа тел, движущихся относительно друг друга. Первый тип – это основное, родительское, тело, второй – тело или тела, совершающие, вообще говоря, достаточно сложные движения относительно основного тела. В такой ситуации трудно, а порою даже невозможно, использовать традиционный подход к описанию обтекаемого объекта граничными узлами или граничными элементами расчётной сетки. Такой подход логично применить только к покоящемуся родительскому телу, в системе расчёта которого моделировать передвижения остальных, более мелких, объектов. 

Для описания движущихся тел в качестве наиболее удобного подхода все большее распространение приобретают методы погруженных границ (в англоязычной научной литературе – Immersed Boundary Conditions или IBC метод)~\cite{mittal2005immersed, boiron2009high, brown2014characteristic}.
Методы погруженных границ составляют целый класс методов, позволяющих обеспечить выполнение граничных условий на поверхности обтекаемых тел в процессе численного расчёта без описания его геометрии сеточными узлами. Это означает, что может быть использована «сплошная» расчётная сетка, покрывающая всю область определения задачи, включая препятствие. 

Как и в случае неподвижных тел, повысить точность моделирования перемещающихся объектов и образующихся вокруг них пограничных слоёв в рамках метода погруженных границ можно за счёт сгущения сетки в областях, прилегающих к погруженным границам таких объектов. Возможным, хотя далеко не оптимальным способом  организации нужного сгущения, является использование стационарных сеток с достаточно мелкими сеточными элементами в подобласти, покрывающей траекторию возможного движения моделируемого тела. Понятно, что для продвижения на сколь-нибудь дальнее расстояние этот метод является весьма трудоёмким, т.к. ведёт к необходимости использования огромных расчётных сеток. К тому же, он не даёт решения в том случае, когда траектория движения тела неизвестна заранее, а определяется в процессе расчёта на основе вычисления действующих на тело аэродинамических сил. 

Выходом в такой ситуации может стать использование динамической адаптации сетки, причём такой, которая была бы заметно дешевле по стоимости расчёта на изначально всюду очень подробных стационарных сетках и не приводила к ухудшению параллельной эффективности всего численного алгоритма. 

В данной работе рассматривается подход, основанный на преобразовании координат узлов или «подвижной сетке» и его адаптация к алгоритму моделирования течения вокруг подвижных твёрдых тел, заданных погруженными границами. Работоспособность разработанных алгоритмов демонстрируется на решении двумерных тестовых задач на неструктурированных треугольных сетках при использовании упрощённой формулировки метода.

{\aim} данной работы является разработка эффективной технологии динамической адаптации неструктурированных сеток к границе движущегося обтекаемого тела заданного методом погруженных граничных условий. Для~достижения поставленной цели необходимо было решить следующие {\tasks}:
\begin{enumerate}
	\item Анализ существующих алгоритмов динамической сеточной адаптации.
  \item  Разработка алгоритма динамической адаптации неструктурированной сетки. 
   \item Адаптация алгоритма к использованию погруженных граничных условий и функции линий уровня, описывающей геометрию твёрдого тела.
  \item Реализация алгоритма в рамках некоммерческого исследовательского комплекса программ NOISEtte++, разрабатываемом в отделе вычислительной аэродинамики и аэроакустики ИПМ РАН.
  \item Рассмотрение двумерной постановки для треугольных сеток. Тестирование на тестовых задачах с аналитически заданной геометрией тела и заданным законом его движения.
\end{enumerate}

{\novelty}
Новаторство предлагаемых в работе решений заключается, в первую очередь в применении метода «быстрой», не меняющей топологии, динамической адаптации сетки в комплексе с заданием геометрии обтекаемого объекта погруженными граничными условиями. 

{\influence} Данный подход к адаптации сетки позволяет создание более эффективной методики численного исследования движущихся в турбулентных потоках тел по сравнению с существующими подходами, использующими неструктурированные сетки.  Помимо использования алгоритма адаптации совместно в погруженными граничными условиями, его можно успешно использоваться для решения задач с «традиционным» описанием подвижных тел граничными сеточными узлами при условии не столь
больших отклонений от их исходного положения (например, разного рода 
колебательных движений), а также, для адаптации сетки к полям решения.

{\methods} Методы работы основаны численном решении дифференциальных уравнений в частных производных. Результаты тестовых расчетов сравниваются с данными эксперимента.

{\defpositions}
\begin{enumerate}
  \item Первое положение
  \item Второе положение
  \item Третье положение
  \item Четвертое положение
\end{enumerate}

{\reliability} разработанных алгоритмов демонстрируется на решении двумерных тестовых задач на неструктурированных треугольных сетках. Полученные результаты сравниваются с экспериментальными данными, полученными другими авторами.

{\probation}
Основные результаты работы докладывались~на:
\begin{enumerate}
	\item Доклад на 6-ой Всероссийская конференция «Вычислительный эксперимент в аэроакустике» (г. Светлогорск Калининградской области, 19-24 сентября 2016 года). Тема доклада: «Моделирование аэродинамики движущихся тем с использованием сеточной адаптации в погруженным границам на неструктурированных треугольных сетках».~\citeauthor{ceaa_2016}
\end{enumerate}

{\publications} Основные результаты по теме диссертации опубликованы в 1 работе в издании, входящем в базу Scopus:
\begin{enumerate}
	\item Статья ``Моделирование аэродинамики движущегося тела, заданного погруженными границами на динамической адаптивной неструктурированной сетке`` в журнале ``Математическое моделирование``~\citeauthor{abalakin2018russian}.\\
	Статья переведена на английский язык: ``Simulating Aerodynanics of a Moving Body Specified by Immersed Boundaries on Dynamically Adaptive Unstructured Meshes`` в журнале ``Mathematical Models and Computer Simulations``~\citeauthor{abalakin2019simulating}
\end{enumerate}
 % Характеристика работы по структуре во введении и в автореферате не отличается (ГОСТ Р 7.0.11, пункты 5.3.1 и 9.2.1), потому её загружаем из одного и того же внешнего файла, предварительно задав форму выделения некоторым параметрам

\textbf{Объем и структура работы.} Диссертация состоит из~введения, трёх глав,
заключения и~двух приложений.
%% на случай ошибок оставляю исходный кусок на месте, закомментированным
%Полный объём диссертации составляет  \ref*{TotPages}~страницу
%с~\totalfigures{}~рисунками и~\totaltables{}~таблицами. Список литературы
%содержит \total{citenum}~наименований.




