\chapter*{Заключение}                       % Заголовок
\addcontentsline{toc}{chapter}{Заключение}  % Добавляем его в оглавление
В статье разрабатывается комплексная технология моделирования течения вязкого сжимаемого газа вокруг движущихся тел в неоднородном потоке. Ключевыми элемента-ми этой технологии является объединение преимуществ метода погруженных границ для описания твёрдых объектов в односвязной области и алгоритма динамической адаптации сетки на остове перераспределения узлов с сохранением сеточной топологии. Также немаловажным обстоятельством при разработке соответствующих методов является обеспечение их эффективной работы на неструктурированных сетках. К настоящему времени данная технология разработана применительно к решению двумерных задач газовой ди-намики на треугольных сетках. Результаты, подтверждающие ее эффективность, демонстрируются на модельных задачах и, в частности, на задаче о моделировании течения за движущимся цилиндром.

В дальнейшем планируется использовать уравнение движения сетки в полной формулировке, а также обобщение разработанной технологии на трёхмерные постановки. Также предполагается работа по усовершенствованию техники адаптации неструктурированной сетки за счёт улучшения качества адаптивной сетки вблизи погруженной границы обтекаемого тела. 

%% Согласно ГОСТ Р 7.0.11-2011:
%% 5.3.3 В заключении диссертации излагают итоги выполненного исследования, рекомендации, перспективы дальнейшей разработки темы.
%% 9.2.3 В заключении автореферата диссертации излагают итоги данного исследования, рекомендации и перспективы дальнейшей разработки темы.
%% Поэтому имеет смысл сделать эту часть общей и загрузить из одного файла в автореферат и в диссертацию:

Основные результаты работы заключаются в следующем.
%% Согласно ГОСТ Р 7.0.11-2011:
%% 5.3.3 В заключении диссертации излагают итоги выполненного исследования, рекомендации, перспективы дальнейшей разработки темы.
%% 9.2.3 В заключении автореферата диссертации излагают итоги данного исследования, рекомендации и перспективы дальнейшей разработки темы.
\begin{enumerate}
	\item На основе анализа существующих методов динамической сеточной адаптации выбран метод адаптации с движением узлов без изменения топологии сетки.
	\item  Алгоритма динамической адаптации неструктурированной сетки разработан для  использования совместно в методом погруженных граничных условий и функции линий уровня, описывающей геометрию твёрдого тела.
	\item Алгоритм реализаван в рамках некоммерческого исследовательского комплекса программ NOISEtte++, разрабатываемом в отделе вычислительной аэродинамики и аэроакустики ИПМ РАН.
	\item Рассмотренна двумерная постановка для треугольных сеток. Проведен численный расчет задаче о моделировании течения за движущимся цилиндром.
\end{enumerate}

И какая-нибудь заключающая фраза.

Последний параграф может включать благодарности.  В заключение автор
выражает благодарность и большую признательность научному руководителю
Козубской~Т.\:К. за поддержку, помощь, обсуждение результатов и научное
руководство. Также автор благодарит Сидорова~А.\:А. и~Петрова~Б.\:Б.
за помощь в~работе с~образцами, Рабиновича~В.\:В. за предоставленные
образцы и~обсуждение результатов, Занудятину~Г.\:Г. и авторов шаблона
*Russian-Phd-LaTeX-Dissertation-Template* за~помощь в оформлении
диссертации. Автор также благодарит много разных людей
и~всех, кто сделал настоящую работу автора возможной.
