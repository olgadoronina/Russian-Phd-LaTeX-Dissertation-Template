\chapter*{Заключение}                       % Заголовок
\addcontentsline{toc}{chapter}{Заключение}  % Добавляем его в оглавление
В работе разрабан упрощенный быстрый алгоритм динамической адаптации сетки на основе перераспределения узлов с сохранением сеточной топологии. Этот метод в комплексе с методом погруженных границ для описания твёрдых объектов в односвязной области позволяет  эффективное численное исследование движущихся в турбулентных потоках тел.  Помимо использования алгоритма адаптации совместно в погруженными граничными условиями, его можно успешно использоваться для решения задач с «традиционным» описанием подвижных тел граничными сеточными узлами при условии не столь
больших отклонений от их исходного положения (например, разного рода 
колебательных движений), а также, для адаптации сетки к полям решения.

Основные результаты работы заключаются в следующем:
%% Согласно ГОСТ Р 7.0.11-2011:
%% 5.3.3 В заключении диссертации излагают итоги выполненного исследования, рекомендации, перспективы дальнейшей разработки темы.
%% 9.2.3 В заключении автореферата диссертации излагают итоги данного исследования, рекомендации и перспективы дальнейшей разработки темы.
\begin{enumerate}
	\item На основе анализа существующих методов динамической сеточной адаптации выбран метод адаптации с движением узлов без изменения топологии сетки.
	\item  Алгоритма динамической адаптации неструктурированной сетки разработан для  использования совместно в методом погруженных граничных условий и функции линий уровня, описывающей геометрию твёрдого тела.
	\item Алгоритм реализаван в рамках некоммерческого исследовательского комплекса программ NOISEtte++, разрабатываемом в отделе вычислительной аэродинамики и аэроакустики ИПМ РАН.
	\item Рассмотренна двумерная постановка для треугольных сеток. Проведен численный расчет задаче о моделировании течения за движущимся цилиндром.
\end{enumerate}


К настоящему времени данная технология разработана применительно к решению двумерных задач газовой динамики на треугольных сетках. Результаты, подтверждающие ее эффективность, продемонстрированны на модельных задачах и, в частности, на задаче о моделировании течения за движущимся цилиндром.

В дальнейшем планируется использовать уравнение движения сетки в полной формулировке, а также обобщение разработанной технологии на трёхмерные постановки. Также предполагается работа по усовершенствованию техники адаптации неструктурированной сетки за счёт улучшения качества адаптивной сетки вблизи погруженной границы обтекаемого тела. 

%% Согласно ГОСТ Р 7.0.11-2011:
%% 5.3.3 В заключении диссертации излагают итоги выполненного исследования, рекомендации, перспективы дальнейшей разработки темы.
%% 9.2.3 В заключении автореферата диссертации излагают итоги данного исследования, рекомендации и перспективы дальнейшей разработки темы.
%% Поэтому имеет смысл сделать эту часть общей и загрузить из одного файла в автореферат и в диссертацию:


В заключение автор
выражает благодарность и большую признательность научному руководителю
Козубской~Т.\:К. за поддержку, помощь, обсуждение результатов и научное
руководство. Также автор благодарит Бахвалава~П.\:А. и~Абалакина~И.\:В.
за помощь в~работе и~обсуждение результатов.
