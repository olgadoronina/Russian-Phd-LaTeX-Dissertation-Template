\chapter{Математическое описание аэродинамики движущихся тел на основе метода по-груженных границ на подвижной сетке} \label{ch:ch1}

\section{Математическая модель для неподвижных сеток} \label{sec:ch1/sec1}

	Следуя методу Бринкмана штрафных функций, система уравнений Навье-Стокса, описывающая течение в области $\Omega_f$ вокруг твёрдого тела $\Omega_B$ с границей $\partial \Omega_B$ , модифицируется следующим образом~\cite{mittal2005immersed, boiron2009high}:
	\begin{equation}
	\left\{
	\begin{aligned}
	&\mathbf{Q}_t + \nabla \cdot \mathbf{F}(\mathbf{Q})+S_{\eta} = \nabla\cdot\mathbf{F}_{\nu}(\nabla \mathbf{Q}), \quad (\mathbf{x}, t) \in \Omega_f \cup \Omega_B \cup \partial\Omega_B\\
	&\mathbf{Q}(t=0) = \mathbf{Q}_0(\mathbf{x})\\
	&S_{\eta} = \eta^{-1} \chi (0, \rho(\mathbf{u} -\mathbf{U}_B), E)
	\end{aligned}
	\right. ,
	\label{eq:NS}
	\end{equation}
	где   $\mathbf{Q} = (\rho, \rho\mathbf{u}, E)$ – вектор безразмерных консервативных переменных,  $\mathbf{F}$ и  $\mathbf{F}_{\nu}$ – конвективные и вязкие потоки, $\rho$ – плотность газа,  $\mathbf{u}$ – вектор скорости, $\mathbf{U}_B$   – скорость движения твёрдого тела,  $E$ – полная энергия. Система замыкается уравнением состояния идеального газа: $p = (\gamma - 1)\rho \varepsilon$, где $\varepsilon$  – внутренняя энергия, $\gamma = 1.4$  – показатель адиабаты, а полная энергия определяется соотношением:  $E = \rho \varepsilon + \rho\mathbf{u}^2/2$.
	
	Характеристическая функция $\chi$ определяет геометрическое место обтекаемого тела:
	\begin{equation}
	\chi = \left\{
	\begin{aligned}
	&1, \quad \mathbf{x}\in \overline{\Omega}_B\\
	&0, \quad \mathbf{x}\in \Omega_f
	\end{aligned}
	\right. .
	\label{eq:chi}
	\end{equation}
	
	Как видно из~\eqref{eq:NS}, источниковые члены $S_\eta$, называемые штрафными функциями, добавляются только в уравнения импульса и энергии. В точках сетки, где характеристическая функция  $\chi\ne0$ (т.е. внутри препятствия), система трансформируется в модель Бринкмана пористой среды с параметром $\eta$, определяющим ее проницаемость: $0<\eta<<1$: чем меньше коэффициент $\eta$, тем точнее воспроизводятся граничные условия при  $\mathbf{x}\in\partial\Omega_B$.

	В работе~\cite{abalakin2015} возможности данного метода были продемонстрированы на примере решения двумерных модельных задач по расчёту дозвукового и сверхзвукового течения вокруг цилиндра.  Работа~\cite{abalakin2018} демонстрирует численное решение задачи турбулентного обтекания трёхмерного цилиндра.
	Для численного решения системы уравнений~\eqref{eq:NS} использовалась оригинальная EBR (Edge-Based Reconstruction) схема повышенной точности, основанная на квазиодномерной рёберно-ориентированной реконструкции переменных~\cite{abalakin2016edge}.
\section{Математическая модель для подвижных сеток}
Для описания течения на движущихся адаптивных (деформируемых) сетках использовался смешанный эйлерово-лагранжев (СЭЛ) подход~\cite{hirt1974arbitrary, golovizin1981} использования уравнений газовой динамики, как законов сохранения. Этот подход можно интерпретировать как промежуточное описание движения сплошной среды между эйлеровым подходом, использующим абсолютную систему координат и лагранжевым описанием, использующим систему координат, связанную с локальной скоростью движения среды. То есть выбирается такая система координат, которая не фиксирована по времени и не перемещается со скоростью среды. 

Рассмотрим систему координат, движущуюся со скоростью $\dot{\mathbf{x}}$, равной скорости деформации сетки $\mathbf{U}_M$. Рассмотрим интеграл от вектора консервативных производных по произвольному жидкому объёму $V(t)$. Тогда полная производная от жидкого объёма запишется следующим образом~\cite{budak2002}, с.259]
\begin{equation}\label{eq:integral_derivative}
\frac{d}{dt}\int \limits_{V(t)}\mathbf{Q}dV = \int \limits_{V(t)}\left[\mathbf{Q}_t + \nabla \cdot (\mathbf{Q}\dot{\mathbf{x}})\right]dV .
\end{equation}
Из законов сохранения~\eqref{eq:NS} следует, что $\mathbf{Q}_t = -\nabla \cdot \mathbf{F}(\mathbf{Q})-S_{\eta} + \nabla\cdot\mathbf{F}_{\nu}(\nabla \mathbf{Q})$. Подставляя частную производную по времени в~\cite{eq:integral_derivative} получаем систему законов сохранения в СЭЛ форме
\begin{equation}\label{eq:integral_derivative2}
\frac{d}{dt}\int \limits_{V(t)}\mathbf{Q}dV = 
-\int \limits_{V(t)}\left[ \nabla \cdot(\mathbf{F} - \mathbf{Q}\dot{\mathbf{x}}) - \nabla\cdot\mathbf{F}_{\nu}(\nabla \mathbf{Q}) + S_{\eta} \right] dV .
\end{equation}
При заданном законе движения сетки  $\dot{\mathbf{x}} = \mathbf{U}_M$ система законов сохранения~\eqref{eq:integral_derivative2} определяет математическую модель необходимую для проведения расчётов на адаптивной сетке с использованием метода погруженных границ. Заметим, что при  $\dot{\mathbf{x}} = 0$ система~\eqref{eq:integral_derivative2} имеет эйлеровскую форму представления законов сохранения, а при $\dot{\mathbf{x}} = \mathbf{u}$  – систему законов сохранения в лагранжевой форме.
\section{Технология расчета системы газодинамических уравнений в СЭЛ форме на движущихся сетках с неизменной топологией для треугольной сетки}
Кратко разработанную технологию можно представить для решения системы~\eqref{eq:integral_derivative2} следующим образом. Для простоты дальнейших выкладок определим вязко-невязкий поток  $\mathbf{F(Q, \nabla Q)} \equiv \mathbf{F(Q)} - \mathbf{F}_{\nu}( \nabla \mathbf{Q})$  и положим равным нулю источниковый член (для дальнейших выкладок он не имеет принципиального значения). Для аппроксимации системы законов сохранения~\eqref{eq:integral_derivative2} будем использовать конечно-объёмный подход, который в равной степени применим как к статическим сеткам, так и к меняющимся по любому непрерывному закону. Определим переменные в узлах неструктурированной сетки и будем рассматривать жидкий контрольный объём $V(t)$  в системе~\eqref{eq:integral_derivative2} как $i$-ую деформируемую сеточную ячейку $K_i(t)$, являющуюся контрольным объёмом, построенным вокруг узла $i$. Тогда для записи конечно-объёмной схемы для системы~\eqref{eq:integral_derivative2} заменим в правой части этой системы интегрирование по объёму интегрированием по поверхности  $\partial K_i(t)$. Получаем
\begin{equation}\label{eq:integral_derivative3}
\frac{d}{dt}\iint \limits_{K_i(t)} \mathbf{Q}dV + 
-\int \limits_{\partial K_i(t)}\mathbf{F(Q, \nabla Q)} \cdot \mathbf{n} \,dS - 
\int \limits_{\partial K_i(t)}\mathbf{Q}(\mathbf{U}_M\cdot \mathbf{n})dS = 0 ,
\end{equation}
где  $\mathbf{n}$ – нормаль к границе ячейки. Таким образом, интегральное уравнение, а, следовательно, и конечно-объёмная разностная схема приобретает дополнительное слагаемое, пропорциональное скорости движения граней сеточных ячеек.

При аппроксимации уравнений Эйлера или Навье-Стокса на движущихся неструктурированных сетках принципиально важным является одновременное обеспечение консервативности (выполнения законов сохранения) и геометрической консервативности (точной передачи однородного потока). Консервативность метода обеспечивается благодаря строгому следованию конечно-объёмному подходу. Для геометрической консервативности  аппроксимация уравнения~\eqref{eq:integral_derivative3} должна быть точной, если  $\mathbf{Q}$ не зависит от координат и времени. Второе слагаемое при этом зануляется, поэтому рассмотрим аппроксимацию третьего слагаемого. Она осуществляется по-разному при применении явных и неявных схем для интегрирования разностного уравнения~\eqref{eq:integral_derivative3}.

В случае многостадийной явной схемы (схемы Рунге-Кутта) необходимо на каждом временном подслое аппроксимировать последнее слагаемое в \eqref{eq:integral_derivative3} так, чтобы при $\mathbf{Q} = const$  аппроксимация оказывалась точной. Для этого будем использовать следующую аппроксимацию на грани двух контрольных объёмов с центрами в точках $R$ и $L$ $(\partial K_L(t)\cap\partial K_R(t) )$:
\begin{equation}\label{eq:approx}
\int \limits_{\partial K_i(t)}\mathbf{Q}(\mathbf{U}_M\cdot \mathbf{n})dS \approx \frac{\mathbf{Q}_L + \mathbf{Q}_R}{2} \int \limits_{\partial K_L(t)\cap\partial K_R(t)} \mathbf{U}_M\cdot \mathbf{n}\,dS,
\end{equation}
а оставшийся интеграл в правой части считать точно на каждый момент времени. В разработанной технологии в течение одного временного шага скорость движения сеточных узлов предполагается постоянной. Если $K_i(t)$  есть барицентрический\footnote{Под барицентрическим контрольным объёмом для треугольной сетки понимается многогранник, границей которого, является ломаная соединяющая центры рёбер и центры тяжести треугольников.}  контрольный объём, то радиус-вектор каждой вершины этого объёма является линейной комбинацией радиус-векторов сеточных узлов. В результате средняя скорость по каждой грани сеточной ячейки также является постоянной в пределах шага, а интеграл в правой части \eqref{eq:integral_derivative3} за счёт изменения площадей граней – линейной функцией времени. Величины контрольных объёмов, следовательно, являются квадратичной функцией. Поэтому, чтобы выполнялась геометрическая консервативность в \eqref{eq:integral_derivative3}, необходимо использовать метод Рунге-Кутта не ниже 2-го порядка.

Для записи неявного метода Эйлера проинтегрируем \eqref{eq:integral_derivative3} по времени. Получаем
\begin{equation}\label{eq:integral_time}
\iint \limits_{K_i(t^{n+1})} \mathbf{Q}dV  - \iint \limits_{K_i(t^{n})} \mathbf{Q}dV+ 
\int \limits_{t^n}^{t^{n+1}} \int \limits_{\partial K_i(t)}\mathbf{F(Q, \nabla Q)} \cdot \mathbf{n} \,dS - 
\int \limits_{t^n}^{t^{n+1}} \int \limits_{\partial K_i(t)}\mathbf{Q}(\mathbf{U}_M\cdot \mathbf{n})dS = 0 ,
\end{equation}
Чтобы аппроксимация была точной при  $\mathbf{Q} = const$, запишем для третьего слагаемого в \eqref{eq:integral_derivative3} аппроксимацию
\begin{equation}\label{eq:approx_time}
\int \limits_{t^n}^{t^{n+1}} \int \limits_{\partial K_i(t)}\mathbf{Q}(\mathbf{U}_M\cdot \mathbf{n})dS \approx \frac{\mathbf{Q}^{n+1}_L + \mathbf{Q}^{n+1}_R}{2} \int \limits_{t^n}^{t^{n+1}} \int \limits_{\partial K_L(t)\cap\partial K_R(t)} \mathbf{U}_M\cdot \mathbf{n}\,dS.
\end{equation}
Интеграл в правой части также вычисляется аналитически.

Аппроксимация второго слагаемого в \eqref{eq:integral_derivative3} для конвективной части системы уравнений Навье-Стокса осуществляется при помощи схем с квазиодномерной реконструкцией переменных, как и в случае недеформируемой сетки. Они основаны на том, что повышение порядка точности осуществляется посредством квазиодномерной рёберно-ориентированной реконструкции значений физических переменных в центр ребра ~\cite{abalakin2016edge}. Коэффициенты реконструкции можно брать с недеформированной сетки или пересчитывать на каждом временном слое. 

Аппроксимация вязких членов в уравнении Навье-Стокса проводится методом Галёркина на основе полиномов 1-го порядка как в случае покоящейся, так и в случае движущейся сетки и не требует внесения изменений. 




