\chapter{Обзор существующих методов динамической адаптации сеток} \label{ch:ch1}

Подробные обзоры методов динамической адаптации сетки представлены в книгах Томпсона~\cite{thompson_handbook_1998}, Лисейкина~\cite{liseikin2009grid}, Хуанг и Рассела~\cite{huang_adaptive_2011}. В целом, все техники адаптации могут быть разделены на две большие группы: $h$-методы которые основаны на локальном измельчении и огрублении сетки, и $r$-методы, которые предполагают передвижение узлов сеточных элементов во всей сетке (или на ее обширном участке), чтобы обеспечить желаемую концентрацию в нужных регионах. В методах локального сеточного измельчения/огрубления ($h$-методы)~\cite{adjerid_high-order_1992,eriksson_adaptive_1991,babuska_p-and_1990} узлы добавляются или, наоборот, удаляются на фиксированных временных слоях так, чтобы численное решение на них разрешалось с заданной точностью. Подход, основанный на преобразовании координат узлов или подвижной сетке во всей расчётной области или в достаточно большой подобласти ($r$-методы)~\cite{huang_adaptive_2011} в основном используется для динамической адаптации сетки к численному решению путём стягивания сеточных узлов в области с большими градиентами. Такая техника успешно используется во всем мире при решении широкого круга прикладных задач и в том числе в вычислительной газовой динамике~\cite{baker_mesh_1997,breslavskii_dynamic_2008,yanenko_methods_1976}. 
Методы подвижной сетки могут быть также разделены на характерные группы алгоритмов, а именно на вариационные методы и методы, основанные на решении дифференциальных уравнений.

	Согласно вариационному подходу, преобразование координат узлов определяется путём минимизации некоторого сеточного функционала. В работах разных авторов предлагаются различные типы подходящих сеточных функционалов. Так, в работе Винслоу (Winslow)~\cite{winslow_adaptive-mesh_1981} представлен эквипотенциальный метод, основанный на диффузии переменных. Вариационный метод Брэкбилла и Зальцмана (Brackbill, Saltzman)~\cite{brackbill_adaptive_1982}  строится на минимизации функции, зависящей от концентрации сеточных узлов, сеточной гладкости и ортогональности. В статье~\cite{dvinsky_adaptive_1991}  предлагается сеточный функционал, зависящий от энергии гармонических отображений, а работах Кнаппа (Knupp) с соавторами \cite{knupp_jacobian-weighted_1996, knupp_framework_2000}  используется идея обусловленности Якобиана преобразования координат. В \cite{huang_variational_2001}  минимизируется функционал, реализующий условия так называемой «равнораспределённости». Наиболее свежий обзор сеточных функционалов, которые могут использоваться с целью адаптации, и их сравнение даётся в работе \cite{huang_comparative_2015}.
	
	В методах, основанных на дифференциальных уравнениях подвижной сетки (MMPDE – moving mesh partial differential equations)~\cite{huang_moving_1994}, уравнение движения сетки и система газодинамических уравнений часто решаются совместно. В \cite{huang_moving_1998, huang_analysis_1997, huang_high_1998} уравнения движения сетки формулируются в зависимости от градиентов моделируемого течения применительно к двумерным задачам. Практические вопросы формулировки уравнений движения сетки и их решения подробно рассматриваются в работах \cite{huang_practical_2001} и \cite{budd_adaptivity_2009}.
	
	Как в вариационных, так и в MMPDE методах расположением сеточных узлов управляет векторная или скалярная функция-монитор. Функция мониторинга обычно строится с целью оценки ошибки численного решения и равномерного распределения её по всем сеточным узлам~\cite{cao_study_1999}.
	
	К классу MMPDE методов можно отнести и несколько более упрощённый подход к адаптации сетки, который предложен Tukovic и Jasak~\cite{jasak_automatic_2006, tukovic_moving_2012} и реализован в открытой библиотеке OpenFOAM~\cite{jasak_dynamic_2010}. В этом способе перемещения узлов сетки определяется уравнением Лапласа с переменной коэффициента диффузии, выполняющего роль функции-монитора. Авторы показали успешное применение этого метода при численном моделировании многофазных потоков.
	
	В целом, MMPDE методы представляются наиболее эффективным подходом к динамической сеточной адаптации в задачах моделирования течений вокруг движущихся тел, если эти тела описываются при помощи метода погруженных границ. Использование погруженных границ и, благодаря этому, проведение расчётов подвижных объектов в односвязных областях открывает возможность расширения области применения MMPDE методов, которые до сих пор использовались в основном для адаптации сетки к полям численного решения, на задачи, требующие адаптации к поверхности движущегося твёрдого тела. 
	
	Привлекательность именно методов, основанные на решении уравнений движения сетки для этого класса задач объясняется несколькими причинами:
	\begin{itemize}
		\item они сохраняют топологию исходной сетки и не требуют интерполяции решения в новые узлы на каждом временном слое;
		\item сохранение сеточной топологии и отсутствие новых узлов позволяют использовать одну и ту же структуру данных и избежать динамического перераспределения узлов по подобластям декомпозиции при расчёте на параллельных вычислительных системах.
	\end{itemize}
	Такой подход к моделированию аэродинамики движущихся тел на неструктурированных сетках позволяет существенно сократить вычислительные затраты.
	
	