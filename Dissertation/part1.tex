\chapter{Математическое описание аэродинамики движущихся тел на основе метода по-груженных границ на подвижной сетке} \label{ch:ch1}

\section{Математическая модель для неподвижных сеток} \label{sec:ch1/sec1}

	Следуя методу Бринкмана штрафных функций, система уравнений Навье-Стокса, описывающая течение в области   вокруг твердого тела   с границей  , модифи-цируется следующим образом [1], [2]:

.	(1)
где   – вектор безразмерных консервативных переменных,   и   – конвек-тивные и вязкие потоки,   – плотность газа,   – вектор скорости,   – скорость движе-ния твердого тела,   – полная энергия. Система замыкается уравнением состояния иде-ального газа: , где   – внутренняя энергия,   – показатель адиабаты, а полная энергия определяется соотношением:  .
Характеристическая функция   определяет геометрическое место обтекаемого тела:
.
Как видно из (1), источниковые члены  , называемые штрафными функциями, до-бавляются только в уравнения импульса и энергии. В точках сетки, где характеристиче-ская функция   (т.е. внутри препятствия), система трансформируется в модель Бринкмана пористой среды с параметром  , определяющим ее проницаемость:  : чем меньше коэффициент  , тем точнее воспроизводятся граничные условия при  .
В работе [29] возможности данного метода были продемонстрированы на примере решения двумерных модельных задач по расчету дозвукового и сверхзвукового течения вокруг цилиндра Для численного решения системы уравнений (1) использовалась ориги-нальная EBR (Edge-Based Reconstruction) схема повышенной точности, основанная на ква-зиодномерной реберно-ориентированной реконструкции переменных [30].

\section{Математическая модель для подвижных сеток}
Для описания течения на движущихся адаптивных (деформируемых) сетках исполь-зовался смешанный эйлерово-лагранжев (СЭЛ) подход [31], [32] использования уравнений газовой динамики, как законов сохранения. Этот подход можно интерпретировать как промежуточное описание движения сплошной среды между эйлеровым подходом, использующим абсолютную систему координат и лагранжевым описанием, использующим систему координат, связанную с локальной скоростью движения среды. То есть выбирается такая система координат, которая не фиксирована по времени и не перемещается со скоростью среды. 
Рассмотрим систему координат, движущуюся со скоростью, равной скорости деформации сетки. Рассмотрим интеграл от вектора консервативных производных по произвольному жидкому объёму. Тогда полная производная от жидкого объёма запишется следующим образом [33, с.259]

.	(2)
Из законов сохранения (1) следует, что  . Подставляя част-ную производную по времени в (2) получаем систему законов сохранения в СЭЛ форме
.	(3)
При заданном законе движения сетки   система законов сохранения (3) определяет математическую модель необходимую для проведения расчетов на адаптивной сетке с ис-пользованием метода погруженных границ. Заметим, что при   система (3) имеет эй-леровскую форму представления законов сохранения, а при   – систему законов со-хранения в лагранжевой форме.

\section{Технология расчета системы газодинамических уравнений в СЭЛ форме на движущихся сетках с неизменной топологией для треугольной сетки}
Кратко разработанную технологию можно представить для решения системы (3) следующим образом. Для простоты дальнейших выкладок определим вязко-невязкий по-ток   и положим равным нулю источниковый член (для даль-нейших выкладок он не имеет принципиального значения). Для аппроксимации системы законов сохранения (3) будем использовать конечно-объёмный подход, который в равной степени применим как к статическим сеткам, так и к меняющимся по любому непрерыв-ному закону. Определим переменные в узлах неструктурированной сетки и будем рас-сматривать жидкий контрольный объём   в системе (3) как i-ую деформируемую се-точную ячейку  , являющуюся контрольным объёмом, построенным вокруг узла i. Тогда для записи конечно-объёмной схемы для системы (3) заменим в правой части этой системы интегрирование по объёму интегрированием по поверхности  . Получаем
,	(4)
где   – нормаль к границе ячейки. Таким образом, интегральное уравнение, а, следова-тельно, и конечно-объёмная разностная схема приобретает дополнительное слагаемое, пропорциональное скорости движения граней сеточных ячеек.
При аппроксимации уравнений Эйлера или Навье-Стокса на движущихся неструкту-рированных сетках принципиально важным является одновременное обеспечение консер-вативности (выполнения законов сохранения) и геометрической консервативности (точной передачи однородного потока). Консервативность метода обеспечивается благодаря стро-гому следованию конечно-объёмному подходу. Для геометрической консервативности уравнение (4) должно аппроксимация должна быть точной, если   не зависит от коорди-нат и времени. Второе слагаемое при этом зануляется, поэтому рассмотрим аппроксима-цию третьего. Она осуществляется по-разному при применении явных и неявных схем для интегрирования разностного уравнения (4).
В случае многостадийной явной схемы (схемы Рунге-Кутта) необходимо на каждом временном подслое аппроксимировать последнее слагаемое в (4) так, чтобы при   аппроксимация оказывалась точной. Для этого будем использовать следующую аппрок-симацию на грани двух контрольных объёмов с центрами в точках R и L ( ):
,
а оставшийся интеграл в правой части считать точно на каждый момент времени. В разра-ботанной технологии в течение одного временного шага скорость движения сеточных уз-лов предполагается постоянной. Если   есть барицентрический  контрольный объём, то радиус-вектор каждой вершины этого объёма является линейной комбинацией радиус-векторов сеточных узлов. В результате средняя скорость по каждой грани сеточной ячей-ки также является постоянной в пределах шага, а интеграл в правой части (4) за счёт изме-нения площадей граней – линейной функцией времени. Величины контрольных объёмов, следовательно, являются квадратичной функцией. Поэтому, чтобы выполнялась геомет-рическая консервативность в (4), необходимо использовать метод Рунге-Кутта не ниже 2-го порядка.
Для записи неявного метода Эйлера проинтегрируем (4) по времени. Получаем
,
Чтобы аппроксимация была точной при  , запишем для третьего слагаемого в (4) аппроксимацию
,
Интеграл в правой части также вычисляется аналитически.
Аппроксимация второго слагаемого в (4) для конвективной части системы уравне-ний Навье-Стокса осуществляется при помощи схем с квазиодномерной реконструкцией переменных, как и в случае недеформируемой сетки. Они основаны на том, что повыше-ние порядка точности осуществляется посредством квазиодномерной рёберно-ориентированной реконструкции значений физических переменных в центр ребра [30]. Ко-эффициенты реконструкции можно брать с недеформированной сетки или пересчитывать на каждом временном слое. 
Аппроксимация вязких членов в уравнении Навье-Стокса проводится методом Га-лёркина на основе полиномов 1-го порядка как случае покоящейся, так и в случае движу-щейся сетки и не требует внесения изменений. 




