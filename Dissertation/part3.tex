\chapter{Численные результаты} \label{ch:ch3}
Применение сеточной адаптации с использованием методики погруженных границ рассматривалось на примере решения двумерной модельной задачи – численного моделирования течения вокруг цилиндрического препятствия при его движении справа налево с дозвуковой скоростью $U_B$ , соответствующей числу Маха равному 0.2. Число Рейнольдса Re рассчитанное по диаметру и скорости движения цилиндра полагалась равным 200. При численном решении данной задачи в качестве характерной скорости выбиралась скорость звука в невозмущенном потоке газа при нормальных условиях. Безразмерный радиус цилиндра $R$  устанавливался равным величине 0.2. Тогда эффективное числа Рейнольдса  Re при котором проводились расчёты имело значение 400.

Рисунок 1: Фрагменты расчётной подробной сетки 1: слева – зона движения цилиндра, справа – окрестность препятствия.


Рисунок 2: Фрагменты расчётной сетки 2: слева – зона движения цилиндра, справа – окрестность препятствия.

Для проведения расчёта движения цилиндра необходимо иметь подробную сетку на всём пути движения цилиндра, во-первых, для корректного описания физических процессов, происходящих при обтекании препятствия, и во-вторых, для более точного описания границы обтекаемого препятствия при использовании метода погруженных границ. Поэтому использование неподвижных (неадаптивных сеток) приводит к резкому возрастанию вычислительной сложности решения такого типа задач. Применение адаптивной сетки, сгущающейся к границе движущегося тела, и обеспечивающее нужное разрешение сетки в следе за телом, должно уменьшить размер сетки, и, как следствие вычислительную стоимость решения. Таким образом, необходимо сравнить качество численного решения, полученного на трёх сетках – подробной, грубой и грубой с адаптацией.

Для этого расчёты проводились на трёх неструктурированных треугольных сетках. Первая сетка (сетка 1) – подробная сетка с числом узлов равным 391628 (Рис. 1), вторая сетка (сетка 2) – сетка с числом узлов 110372, приблизительно в 3.5 раза меньшим, чем у подробной сетки (Рис. 2). Третья сетка (сетка 3) это сетка 2 адаптирующаяся к поверхности движущегося цилиндрического препятствия и не меняющая своей топологии. На рисунках 3 и 4 показаны фрагменты адаптивной сетки на различные моменты времени положения движущегося цилиндра. Заметим, что в области движения цилиндра для сеток 1 и 2 определён регион квазиравномерной сетки с шагом $h$  и от него происходит огрубление сетки к границам расчётной области (см. рисунок 3 слева вверху)


Рисунок 3: Сетка 2 – недеформированная (слева вверху) и её адаптация на три момента времени: начало периода  , полтора периода   и два периода  .



Рисунок 4: 	Сетка вблизи препятствия на начало периода   и четыре с половиной периода  

Закон адаптации сетки определяется управляющей функцией  $\rho(\mathbf{x})$ в уравнении~\eqref{eq:jasak}. Так как обтекаемое тело симметрично, то достаточно определить зависимость управляющей функции только от радиального направления  $r$ с центром, совпадающим с центром цилиндра. Данная функция должна определять сгущение точек с внешней и внутренней стороны границы цилиндра и не изменять исходную сетку при удалении от центра цилиндрического препятствия. Этим условиям удовлетворяют функции типа гауссиана с управляющими параметрами
\begin{equation}\label{eq:gaussian}
\rho{r} = \left\{
\begin{aligned}
&A \exp{\left[-\frac{\ln 2}{b_1}(r - R)^{n_1}\right]+1}, \quad r\le R,\\
&A \exp{\left[-\frac{\ln 2}{b_2}(r - R)^{n_2}\right]+1}, \quad r > R.\\
\end{aligned}
\right.
\end{equation}
Здесь амплитуда  $A$ и чётные показатели степени $n_1, n_2$,   определяют величину сгущения, значения полуширины гауссиана $b_1, b_2$,   задают толщины областей сгущение внутри и вне цилиндра, соответственно. В настоящем расчёте эти параметры полагались равными следующим значениям
\begin{equation}\label{eq:values}
A = 9, \quad n_1 = n_2 = 4, \quad b_1 = 1.6\cdot 10^{-7}, \quad b_2 = 0.1
\end{equation}
Функция~\eqref{eq:gaussian} с параметрами~\eqref{eq:values} трижды непрерывно дифференцируема и её график приведён на рисунке 5.


Рисунок5: График управляющей функции  .

Из этого графика можно видеть, что наиболее сильная адаптация сетки (уменьшение амплитуды $A$ на 46\%) к границе цилиндра наблюдается на расстоянии приблизительно равным $0.08R$   от границы цилиндра во внутренней области и на расстоянии $3R$ во внешней области. На расстояниях больших $3R$  управляющая функция экспоненциально понижается до единицы, что исключает адаптацию сетки вдали от центра цилиндра в направлении движения цилиндра. Такое изменение сетки можно наблюдать на рисунке 3. Из распределения узлов сетки, показанных на рисунке 3 видно, что в направлении перпендикулярном движению сетка сильно деформируется. Это явление связано с тем, что число узлов в адаптивной сетке не увеличивается, а в направлении перпендикулярном движению исходная недеформированная сетка 2 (слева вверху на рисунке 3) имеет меньшую плотность распределения узлов.

Из рисунков 3 и 4 видно, что адаптация сетки происходит практически однородно и не зависит от положения цилиндра в абсолютной системе координат. То есть фактически можно считать, что деформированная в начальный момент времени сетка перемещается как жёсткая конструкция в направлении движения цилиндра.

Также обратим внимание (см. рисунок 4), что при использовании изотропной управляющей функции (зависимость только от радиуса) не приводит к сильной деформации элементов сетки в области движения цилиндрического препятствия и сетка вне узкой окрестности вдоль границы цилиндра близка к квазиравномерной.

В процессе адаптации грубой сетки 2 размер элементов сетки в окрестности границы цилиндра имеет размер в среднем в 2.5 раза меньший, чем размер аналогичных элементов подробной сетки 1. Это можно видеть на рисунке 6, где показано распределение минимальных высот треугольников на подробной сетке (слева) и на грубой сетке после адаптации (сетка 3). 


Рисунок 6: 	Распределения минимальных высот треугольников в расчётных сетках: по-дробная сетка (слева), адаптированная грубая сетка (справа)

При выполнении расчётов движущегося цилиндра на всех трёх сетках шаг интегрирования по времени выбирался согласно следующему критерию
\begin{equation}\label{eq:time_step}
\Delta t  =\min(\Delta t_{CFD}, \Delta t_{MOV})
\end{equation}
где   $\Delta t_{CFD}$ – шаг определяемый устойчивостью разностной схемы и обеспечивающий корректность нестационарного интегрирования,   $\Delta t_{MOV}$ – временной шаг, выбираемый из условия прохождения приграничных ячеек границей цилиндра не менее чем за один шаг при расчёте временного слоя  $t_1 = t_0+\Delta t$ 
.\begin{equation}\label{eq:time_step_mov}
\Delta t_{MOV} < \frac{h_{\mathrm{min}}}{|U_B|}.
\end{equation}
Здесь  $h_{\mathrm{min}}$ минимальный размер приграничного элемента сетки на время  $t_0$.



Рисунок 7: 	Распределение завихренности на четыре момента времени: начало периода  , полто-ра периода  , два периода  и четыре с половиной периода  


Общая картина течения после выполнения расчёта с использованием адаптивной сетки показана на рисунке 7, где приведены поля завихренности на различные моменты времени. В табл.~\ref{tabl:table} приведены средние значение коэффициента сопротивления  и числа Струхаля Sh, полученные по результатам этих расчетов на трёх сетках и экспериментальные значения из работы~\cite{henderson1997nonlinear}. Число Струхаля  Sh определено с помощью преобразования Фурье.


\begin{table}[]
	\begin{tabular}{|c|c|c|c|c|}
	\hline
	& Эксперимент~\cite{henderson1997nonlinear} & сетка 1 & сетка 2 & сетка 3 \\ \hline
	Sh               & 0.19        & 0.189   & 0.186   & 0.191   \\ \hline
	$\overline{C}_d$ & 1.34        & 1.218   & 1.118   & 1.301   \\ \hline
	\end{tabular}
	\caption{Средние значения коэффициентов силы сопротивления и число Струхаля, полученные расчётом на разных сетках.}
	\label{tabl:table}
\end{table}
\clearpage